First install the express generator:
\begin{verbatim}
  npm init -y
  npm install sequelize mysql2 express @babel/core @babel/cli @babel/node body-parser sequelize-cli --save
  npm install @babel/preset-env nodemon --save-dev
\end{verbatim}

In order to use EcmaScript6 there needs to be a \texttt{.babelrc} file with:
\begin{verbatim}
  {
    "presets": ["@babel/preset-env"]
  }
\end{verbatim}

Node js obviously works very well with JSON. To define our modules we create \texttt{modules.json}
\begin{verbatim}
  ["employees", "shifts"]
\end{verbatim}

First the database connection will be created using sequelize in \texttt{db.js}
\begin{verbatim}
  import modules from "./modules.json";
  import path from "path";
  import Sequelize from "sequelize";

  const db = {};

  const sequelize = new Sequelize("express", "root", "root", {
    host: "127.0.0.1",
    dialect: "mysql",
    operatorsAliases: false
  });

  modules.forEach(module => {
    const api = require("./" + module + "/api.js").default;

    const model = sequelize.import(path.join(__dirname, module, api.modelPath()));
    db[model.name] = model;
  });

  Object.keys(db).forEach(modelName => {
    if (db[modelName].associate) {
      db[modelName].associate(db);
    }
  });

  db.sequelize = sequelize;
  db.Sequelize = Sequelize;

  export default db;
\end{verbatim}

Now the entry file can be created in \texttt{index.js}
\begin{verbatim}
  import express from "express";
  import db from "./db";
  import modules from "./modules";
  import bodyParser from "body-parser";

  var app = express();

  app.use(bodyParser.json());

  modules.forEach(module => {
    require(`./${module}/api`).default.views(app, db);
  });

  app.listen(8000, () => {
    db.sequelize.sync();
  });
\end{verbatim}

Now the api definition in \texttt{api.js}
\begin{verbatim}
  export default class Api {
    static modelPath() {
      return "model.js";
    }

    static views(app) {
      console.error("Add views to the api");
    }
  }
\end{verbatim}

Next up is creating the employee module. First up is the model. First we create the migration by running
\begin{verbatim}
  sequelize migration:create --name create_employee_table
\end{verbatim}

Now there should be a file in the \texttt{migrations} folder ending with \texttt{create\_employee\_table}. Update this file with
\begin{verbatim}
  'use strict';
  module.exports = {
    up: (queryInterface, Sequelize) => {
      return queryInterface.createTable('Employees', {
        id: {
          allowNull: false,
          autoIncrement: true,
          primaryKey: true,
          type: Sequelize.INTEGER
        },
        name: {
          type: Sequelize.STRING
        },
        birth_date: {
          type: Sequelize.DATE
        },
        email: {
          type: Sequelize.STRING
        },
        createdAt: {
          allowNull: false,
          type: Sequelize.DATE
        },
        updatedAt: {
          allowNull: false,
          type: Sequelize.DATE
        }
      });
    },
    down: (queryInterface, Sequelize) => {
      return queryInterface.dropTable('Employees');
    }
  };
\end{verbatim}

Next the model can be created in \texttt{employees/model.js}
\begin{verbatim}
  export default (sequelize, DataTypes) => {
    const Employee = sequelize.define(
      "Employee",
      {
        name: DataTypes.STRING,
        birth_date: DataTypes.DATE,
        email: DataTypes.STRING
      },
      {}
    );

    return Employee;
  };
\end{verbatim}

And to finish the employee api create the file \texttt{employees/api.js}
\begin{verbatim}
  import Api from "../api";

  const prefix = "/employees/";

  export default class EmployeeApi extends Api {
    static views(app, db) {
      const Employee = db.Employee;

      app.post(prefix, function(req, res) {
        Employee.create(req.body).then(employee => {
          res.json(employee);
        });
      });

      app.get(prefix, function(req, res) {
        Employee.findAll().then(employees => {
          res.json(employees);
        });
      });

      app.get(prefix + ":id/", function(req, res) {
        Employee.findByPk(req.params.id).then(employee => {
          res.json(employee);
        });
      });
    }
  }
\end{verbatim}

Next is the shift module. Again create the migrations:
\begin{verbatim}
  sequelize migration:create --name create_shift_table
\end{verbatim}

Now fill the file in the \texttt{migrations} folder ending with \texttt{create\_shift\_table} with
\begin{verbatim}
  "use strict";

  module.exports = {
    up: (queryInterface, Sequelize) => {
      return queryInterface.createTable("Shifts", {
        id: {
          allowNull: false,
          autoIncrement: true,
          primaryKey: true,
          type: Sequelize.INTEGER
        },
        title: {
          type: Sequelize.STRING
        },
        start: {
          type: Sequelize.STRING
        },
        end: {
          type: Sequelize.DATE
        }
      });
    },

    down: (queryInterface, Sequelize) => {
      return queryInterface.dropTable("Shifts");
    }
  };
\end{verbatim}

Now create the model in \texttt{shifts/model.js}
\begin{verbatim}
  export default (sequelize, DataTypes) => {
    const Shift = sequelize.define(
      "Shift",
      {
        title: DataTypes.STRING,
        start: DataTypes.STRING,
        end: DataTypes.DATE
      },
      {}
    );

    Shift.associate = models => {
      models.Shift.belongsToMany(models.Employee, {
        onDelete: "CASCADE",
        through: "ShiftEmployee"
      });
    };

    return Shift;
  };

\end{verbatim}

And finish it by creating the api in \texttt{shifts/api.js}
\begin{verbatim}
  import Api from "../api";

  const prefix = "/shifts/";

  export default class ShiftApi extends Api {
    static views(app, db) {
      const Shift = db.Shift;

      const withEmployee = {
        include: [
          {
            model: db.Employee,
            through: "ShiftEmployee"
          }
        ]
      };

      app.post(prefix, (req, res) => {
        Shift.create(req.body).then(async shift => {
          await shift.setEmployees(req.body.employees);
          res.json(shift);
        });
      });

      app.get(prefix, function(req, res) {
        Shift.findAll(withEmployee).then(shifts => {
          res.json(shifts);
        });
      });

      app.get(prefix + ":id/", function(req, res) {
        Shift.findByPk(req.params.id, withEmployee).then(shift => {
          res.json(shift);
        });
      });
    }
  }
\end{verbatim}

Add the start script to package.json
\begin{verbatim}
  {
    ...
    "scripts": {
      "start": "nodemon index.js --watch server --exec babel-node"
    }
  }
\end{verbatim}

Now run \texttt{npm start} in your terminal
