The versions used are:
\begin{itemize}
  \item php 7.2.11
  \item composer 1.8.5
  \item laravel 5.8.19
\end{itemize}

First install laravel via composer
\begin{verbatim}
  composer global require laravel/installer
\end{verbatim}

Then create the laravel project
\begin{verbatim}
  laravel new modular_monolith
\end{verbatim}

Unlike some other frameworks you need to create the migrations yourself by running:
\begin{verbatim}
  php artisan make:migration create_employees_table
  php artisan make:migration create_shifts_table
  php artisan make:migration create_employee_shift_table
\end{verbatim}

There will be a file created in \texttt{database/migrations} which ends with \texttt{create\_employees\_table}. Replace the \texttt{up} function with this:
\begin{verbatim}
  Schema::create('employee', function (Blueprint $table) {
    $table->increments('id');
    $table->string( 'name');
    $table->date('birth_date');
    $table->string('email');
  });
\end{verbatim}

And for the file that ends with \texttt{create\_shifts\_table}
\begin{verbatim}
  Schema::create('shifts', function (Blueprint $table) {
    $table->increments('id');
    $table->string('title');
    $table->date('start');
    $table->date('end');
  });
\end{verbatim}

Now last of all the table that connects the two models needs to be made. This is because it is a many to many relation. Laravel does not pick this us automatically. Therefore again replace the \texttt{up} function of the file that ends with \texttt{create\_employee\_shift\_table}
\begin{verbatim}
  Schema::create('employee_shift', function (Blueprint $table) {
      $table->bigIncrements('id');

      $table->integer('employee_id')->unsigned()->nullable();
      $table->foreign('employee_id')->references('id')
          ->on('employees')->onDelete('cascade');

      $table->integer('shift_id')->unsigned()->nullable();
      $table->foreign('shift_id')->references('id')
          ->on('shifts')->onDelete('cascade');
  });
\end{verbatim}

First the \texttt{.env} needs to know the database settings. Replace these variables
\begin{verbatim}
  DB_DATABASE=laravel
  DB_USERNAME=root
  DB_PASSWORD=root
\end{verbatim}

Now migrate the database by running
\begin{verbatim}
  php artisan migrate
\end{verbatim}

Then create the Employee model in \texttt{app/Employees/Employee.php}
\begin{verbatim}
  <?php
  namespace App\Employees;

  use Illuminate\Database\Eloquent\Model;

  class Employee extends Model
  {
      public $name;
      public $birth_date;
      public $email;
      public $fillable = ['name', 'birth_date', 'email'];
      public $timestamps = false;

      public function shifts()
      {
          return $this->belongsToMany(\App\Shifts\Api::getModel());
      }
  }
\end{verbatim}

Then the controller for Employees will be created:
\begin{verbatim}
  <?php
  namespace App\Employees;

  use App\Http\Controllers\Controller;
  use Illuminate\Http\Request;

  class EmployeeController extends Controller
  {
      public function index()
      {
          return response()->json(Employee::with('shifts')->get());
      }

      public function store(Request $request)
      {
          $employee = $request->all();
          $employee['birth_date'] = \Carbon\Carbon::parse($employee['birth_date']);

          return Employee::create($employee);
      }

      public function show($id)
      {
          return Employee::with('shifts')->find($id);
      }
  }
\end{verbatim}

Next is the shift model
\begin{verbatim}
  <?php
  namespace App\Shifts;

  use Illuminate\Database\Eloquent\Model;

  class Shift extends Model
  {
      public $title;
      public $start;
      public $end;

      public $fillable = ['title', 'start', 'end'];
      public $timestamps = false;

      public function employees()
      {
          return $this->belongsToMany(\App\Employees\Api::getModel());
      }
  }
\end{verbatim}

And the matching Controller:
\begin{verbatim}
  <?php
  namespace App\Shifts;

  use App\Http\Controllers\Controller;
  use Illuminate\Http\Request;

  class ShiftController extends Controller
  {
      public function index()
      {
          return response()->json(Shift::with(['employees'])->get());
      }

      public function store(Request $request)
      {
          $shiftArray = $request->all();
          $shiftArray['start'] = \Carbon\Carbon::parse($shiftArray['start']);
          $shiftArray['end'] = \Carbon\Carbon::parse($shiftArray['end']);

          $shift = Shift::create($shiftArray);

          $shift->employees()->sync($shiftArray['employees']);

          return $shift;
      }

      public function show($id)
      {
          return Shift::with(['employees'])->find($id);
      }
  }
\end{verbatim}

Now the urls can be mapped to the controller by adding this to the \texttt{url/web.php}
\begin{verbatim}
  Route::resources([
    'employees' => '\App\Employees\EmployeeController',
    'shifts' => '\App\Shifts\ShiftController'
  ]);
\end{verbatim}

The last thing to do is disable csrf for the paths. This can be done in \texttt{app/Http/Middelware/VerifyCsrfToken.php}
\begin{verbatim}
  protected $except = [
    'employees',
    'shifts'
  ];
\end{verbatim}

Now run the app with
\begin{verbatim}
  php artisan serve
\end{verbatim}
