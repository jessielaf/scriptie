The versions used are:
\begin{itemize}
  \item React 16.8.6
  \item Node 10.15.3
  \item npm 6.4.1
\end{itemize}

First thing to do is start the react app by running:
\begin{verbatim}
  npx create-react-app modular_monolith
\end{verbatim}

React does not come with it's own router. Thus you have to install one. The most used is react router. You can install this by running
\begin{verbatim}
  npm install --save react-router-dom
\end{verbatim}

React does not have it's own HTTP client. So it needs to be installed:
\begin{verbatim}
  npm install --save axios
\end{verbatim} 

React also does not have its own package so we can install \texttt{http-proxy-middleware}
\begin{verbatim}
  npm install --save http-proxy-middleware
\end{verbatim}

And create the proxy in \texttt{src/setupProxy.js} and paste this init:
\begin{verbatim}
  const proxy = require("http-proxy-middleware");

  module.exports = function(app) {
    app.use(
      proxy("/api", {
        target: "http://localhost:8000",
        pathRewrite: {
          "^/api": ""
        }
      })
    );
  };
\end{verbatim}

Now create the api between modules in \texttt{src/api.js}
\begin{verbatim}
  export default class Api {
    static create(object) {
      console.log(object, "not saved");
      console.error("Implement the create functionality");
    }

    static retrieve(id) {
      console.log(id, "not retrieved");
      console.error("Implement the retrieve functionality");
    }

    static overview() {
      console.error("Implement the overview functionality");
    }

    static route() {
      console.error("Returns the route for the router");
    }
  }
\end{verbatim}

Now the first create the employee. Start with the overview:

src/employees/Overview.js
\begin{verbatim}
  import React from "react";
  import { Link } from "react-router-dom";
  import EmployeeApi from "./api";

  class Overview extends React.Component {
    constructor() {
      super();
      this.state = {
        employees: []
      };
    }

    componentDidMount() {
      EmployeeApi.overview().then(response => {
        this.setState({
          employees: response.data
        });
      });
    }

    render() {
      const employeesView = [];

      this.state.employees.forEach(employee => {
        employeesView.push(
          <li key={employee.id}>
            <Link to={"/employee/" + employee.id}>{employee.name}</Link>
          </li>
        );
      });

      return (
        <div>
          <ul>{employeesView}</ul>
          <Link to="/employees/create">Create employee</Link>
        </div>
      );
    }
  }

  export default Overview;
\end{verbatim}

Now the creation form:

src/employees/Create.js
\begin{verbatim}
  import React from "react";
  import EmployeeApi from "./api";

  class Overview extends React.Component {
    constructor() {
      super();
      this.state = {
        employee: {
          name: "",
          birth_date: "",
          email: ""
        }
      };
    }

    handleInputChange(event) {
      const target = event.target;
      const employee = this.state.employee;
      employee[target.name] = target.value;

      this.setState({
        employee: employee
      });
    }

    submit(event) {
      event.preventDefault();
      EmployeeApi.create(this.state.employee).then(() => {
        this.props.history.push("/employees");
      });
    }

    render() {
      return (
        <form onSubmit={this.submit.bind(this)}>
          <div>
            <label htmlFor="name">Name: </label>
            <input
              id="name"
              value={this.state.employee.name}
              onChange={this.handleInputChange.bind(this)}
              placeholder="Name"
              type="text"
              name="name"
            />
          </div>

          <div>
            <label htmlFor="birthDate">Birth date: </label>
            <input
              id="birthDate"
              value={this.state.employee.birth_date}
              onChange={this.handleInputChange.bind(this)}
              placeholder="01-04-1998"
              type="text"
              name="birth_date"
            />
          </div>

          <div>
            <label htmlFor="email">Email: </label>
            <input
              id="email"
              value={this.state.employee.email}
              onChange={this.handleInputChange.bind(this)}
              placeholder="jessie@example.com"
              type="email"
              name="email"
            />
          </div>

          <button type="submit">Submit</button>
        </form>
      );
    }
  }

  export default Overview;
\end{verbatim}

And last of all the detail view:

src/employees/Detail.js
\begin{verbatim}
  import React from "react";
  import EmployeeApi from "./api";

  class Overview extends React.Component {
    constructor(props) {
      super(props);
      this.state = {
        employee: {}
      };
    }

    componentDidMount() {
      EmployeeApi.retrieve(this.props.match.params.id).then(response => {
        this.setState({
          employee: response.data
        });
      });
    }

    render() {
      return (
        <div>
          {this.state.employee.name}

          <div>
            Shifts:
            <ul>
              {this.state.employee.shifts &&
                this.state.employee.shifts.map(shift => (
                  <li key={shift.id}>{shift.title}</li>
                ))}
            </ul>
          </div>
        </div>
      );
    }
  }

  export default Overview;
\end{verbatim}

So when creating the route the need was to create three urls:
\begin{itemize}
  \item /employees/ as overview
  \item /employees/create/ which is the create url
  \item /employees/:id/ which is the detail view
\end{itemize}

But the problem was that the detail view and the create view would merge. This is because the detail is seeing create as the id. Thus the url \texttt{/employees/:id/} to \texttt{/employee/:id}. This is not expected from a framework.

Now create the shift module

src/shifts/Overview.js
\begin{verbatim}
  import React from "react";
  import { Link } from "react-router-dom";
  import shiftApi from "./api";

  class Overview extends React.Component {
    constructor() {
      super();
      this.state = {
        shifts: []
      };
    }

    componentDidMount() {
      shiftApi.overview().then(response => {
        this.setState({
          shifts: response.data
        });
      });
    }

    render() {
      const shiftsView = [];

      this.state.shifts.forEach(shift => {
        shiftsView.push(
          <li key={shift.id}>
            <Link to={"/shift/" + shift.id}>{shift.title}</Link>
          </li>
        );
      });

      return (
        <div>
          <ul>{shiftsView}</ul>
          <Link to="/shifts/create">Create shift</Link>
        </div>
      );
    }
  }

  export default Overview;
\end{verbatim}

src/shifts/Create.js
\begin{verbatim}
  import React from "react";
  import shiftApi from "./api";
  import EmployeeApi from "../employees/api";

  class Overview extends React.Component {
    constructor() {
      super();
      this.state = {
        shift: {
          title: "",
          start: "",
          end: "",
          employees: []
        },
        employees: []
      };
    }

    componentDidMount() {
      EmployeeApi.overview().then(response => {
        this.setState({
          employees: response.data
        });
      });
    }

    handleInputChange(event) {
      const target = event.target;
      const shift = this.state.shift;
      shift[target.name] = target.value;

      this.setState({
        shift: shift
      });
    }

    handleMultiSelect(event) {
      var options = event.target.options;
      const values = [];

      for (var i = 0, l = options.length; i < l; i++) {
        if (options[i].selected) {
          values.push(options[i].value);
        }
      }

      const shift = this.state.shift;
      shift[event.target.name] = values;

      this.setState({
        shift: shift
      });
    }

    submit(event) {
      event.preventDefault();
      shiftApi.create(this.state.shift).then(() => {
        this.props.history.push("/shifts");
      });
    }

    render() {
      return (
        <form onSubmit={this.submit.bind(this)}>
          <div>
            <label htmlFor="title">Title: </label>
            <input
              id="title"
              value={this.state.shift.title}
              onChange={this.handleInputChange.bind(this)}
              placeholder="Title"
              type="text"
              name="title"
            />
          </div>

          <div>
            <label htmlFor="start">Start: </label>
            <input
              id="start"
              value={this.state.shift.start}
              onChange={this.handleInputChange.bind(this)}
              placeholder="01-01-2019"
              type="text"
              name="start"
            />
          </div>

          <div>
            <label htmlFor="end">End: </label>
            <input
              id="end"
              value={this.state.shift.end}
              onChange={this.handleInputChange.bind(this)}
              placeholder="01-01-2019"
              type="text"
              name="end"
            />
          </div>

          <div>
            <label htmlFor="employees">Employees: </label>
            <select
              id="employees"
              value={this.state.shift.employees}
              onChange={this.handleMultiSelect.bind(this)}
              multiple={true}
              name="employees"
            >
              {this.state.employees.map(employee => {
                return (
                  <option key={employee.id} value={employee.id}>
                    {employee.name}
                  </option>
                );
              })}
            </select>
          </div>

          <button type="submit">Submit</button>
        </form>
      );
    }
  }

  export default Overview;
\end{verbatim}

The unfortunate thing that was clear when working with react is that it was needed to create a handler for each different input type.

src/shifts/Detail.js
\begin{verbatim}
  import React from "react";
  import shiftApi from "./api";

  class Overview extends React.Component {
    constructor(props) {
      super(props);
      this.state = {
        shift: {}
      };
    }

    componentDidMount() {
      shiftApi.retrieve(this.props.match.params.id).then(response => {
        this.setState({
          shift: response.data
        });
      });
    }

    render() {
      return (
        <div>
          {this.state.shift.name}

          <div>
            Shifts:
            <ul>
              {this.state.shift.employees &&
                this.state.shift.employees.map(employee => (
                  <li key={employee.id}>{employee.name}</li>
                ))}
            </ul>
          </div>
        </div>
      );
    }
  }

  export default Overview;
\end{verbatim}

In the end there needs to be an replacement in \texttt{src/App.js} 
\begin{verbatim}
  import React from "react";
  import { BrowserRouter as Router, Link } from "react-router-dom";
  import EmployeeApi from "./employees/api";
  import ShiftApi from "./shifts/api";

  function AppRouter() {
    const shiftStyle = {
      marginLeft: "10px"
    };

    return (
      <Router>
        <div>
          <Link to="/employees">Employees</Link>
          <Link to="/shifts/" style={shiftStyle}>
            Shifts
          </Link>

          {EmployeeApi.route()}
          {ShiftApi.route()}
        </div>
      </Router>
    );
  }

  export default AppRouter;
\end{verbatim}

Now the application is able to run with the command:
\begin{verbatim}
  npm start
\end{verbatim}
