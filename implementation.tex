\chapter{Implementation of the architecture}

The chosen implementation is modular monolith. As mentioned in \fullref{sec:ModularMonolith} a modular monolith is based on domain driven design where the modules can be developed separately. Most of the principles can be taken from domain driven design. But in a modular monolith the modules do not know what other modules contain. This is possible over an API.

\section{Characteristics}

Domain driven design was coined by Eric Evans in his book Domain driven design \cite{domainDrivenDesign}. Eric Evans himself said that there is no real standard for domain driven design but there is one for a domain:

\largequote{A domain is a field of study that defines a set of common requirements, terminology, and functionality for any software program constructed to solve a problem in the area of computer programming, known as domain engineering. The word domain is also taken as a synonym of application domain It is also seen as a sphere of knowledge \cite{domainDefinition}}

What is important to note is that each module is linked to a domain but there is a distinct difference between only implementing domain driven design and modular monolith. A modular monolith is part of the software architecture of the application while domain driven design part is of the software design. The difference between those two is that software architecture about converting characteristics or quality attributes into a structured solution where as software design is more about the responsibility each module or section, inside the architecture, has. \cite{softwareArchitectureDefinition}

\section{Current situation}

The table below is to paint a good picture of the current application, situation and map which domains there are inside the EFFE application and which functionalities they possess.

\begin{tabularx}{\linewidth}{|>{}X|>{}X|>{}X|}
    \hline

    Domain
     &
    Functionalities
    \\ \hline

    User
     &
    \begin{compactitem}
        \item Reset password
        \item User CRUD*
    \end{compactitem}
    \\ \hline

    Shift
     &
    \begin{compactitem}
        \item Shift overview
        \item Create shift
    \end{compactitem}
    \\ \hline

    Skill
     &
    \begin{compactitem}
        \item Skill CRUD*
    \end{compactitem}
    \\ \hline

    Store
     &
    \begin{compactitem}
        \item Store CRUD*
    \end{compactitem}
    \\ \hline

    Client
     &
    \begin{compactitem}
        \item Client CRUD*
    \end{compactitem}
    \\ \hline

    Authentication
     &
    \begin{compactitem}
        \item Login
        \item Reset password
    \end{compactitem}
    \\ \hline

    Schedule
     &
    \begin{compactitem}
        \item Generate schedule
    \end{compactitem}
    \\ \hline

    Hour registration
     &
    \begin{compactitem}
        \item Hour registration
    \end{compactitem}
    \\ \hline

    Shift market
     &
    \begin{compactitem}
        \item Shift market
    \end{compactitem}
    \\ \hline

    Shift change
     &
    \begin{compactitem}
        \item Switching shifts
        \item Calling in sick
    \end{compactitem}
    \\ \hline

    Company
     &
    \begin{compactitem}
        \item Managing company settings
    \end{compactitem}
    \\ \hline
\end{tabularx}

\small{\textcolor{gray}{* CRUD or Create Read Update Delete refers to the actions that can be called on an object via the rest API}}

An example of a use case where it is needed to replace one of these modules is if a big client comes to EFFE but requests something small that should be different in the user module. With a modular monolith we can create this new module and place it inside the application and it will work the same as the normal application.

\section{API}
\label{sec:API}

When creating this API the assumption is done that a modern ORM(Object relational mapping) is used.

\largequote{Object-relational-mapping is the idea of being able to write queries, as well as much more complicated ones, using the object-oriented paradigm of your preferred programming language. \cite{ormDefinition}}

This assumption is done because almost every modern framework uses this concept to map objects to a relational database which is what EFFE uses.

Therefore the first attribute defined in our api is the \textbf{model} itself.

Microservices talk with each other via a protocol. The most used protocols are HTTP, TCP or AMQP \cite{microservicesAPI}. What all of these protocols have in common is that they return a serialized version of the response. Most of the time in JSON.

Commonly in web frameworks there is something used called a dataclass or a serializer. This class can convert an object into JSON or any other content type. Thus if the api of a module in the modular monolith can expose such a serializer the application can serialize all the foreign keys the module's model has. But when working with the application of EFFE the company found that each user role may require a other specific serializer. For example: EFFE has three roles: the employment agency employee, the client and the temp worker. If the client and the temp worker want to retrieve the shifts the client also gets the users in that shift while the temp worker only sees the general data of the shift. This is so that temp workers won't have the biased in taking shifts with people they like or vise versa.

Therefore it is important to note that each role should have a specific serializer. The API will have \textbf{base serializer} and the option to change the \textbf{serializer by role}.

\section{Programming language and Web framework}

A web framework is:

\largequote{A software tool that provides a way to build and run web applications. As a result, you don’t need to write code on your own and waste time looking for possible miscalculations and bugs. \cite{webFrameworkDefinition}}

It is a industry standard to use web frameworks. It simply makes life easier. But which web framework? The comparison will be of the front- and backend frameworks and languages. There will not be a research of the whole framework or language. The focus of the research will lay on the compatibility of the framework with the modular monolith architecture.

The reason the focus lays on the web framework is because the modularity mostly comes from the framework. Most popular programming languages have the same capabilities when used right. The framework decides how the language is used and also how it reacts to modularity.

\subsection{Backend}
\label{sec:BackendImplementation}

Even though the programming language is important most of the time their modularity comes from the framework that is implemented. According to hackers.io these are the top backend frameworks in 2019 \cite{topFrameworks}

\begin{enumerate}
    \item Express (Node.js)
    \item Django (Python)
    \item Rails (Ruby)
    \item Laravel (PHP)
    \item Spring (Java)
\end{enumerate}

All frameworks will be tested against the same use case:

The first domain is employees. An employee has:
\begin{itemize}
    \item Name
    \item Birth date
    \item Email
\end{itemize}

The second domain is shifts. A shift has four attributes
\begin{itemize}
    \item Title
    \item Start date
    \item End date
    \item Employees
\end{itemize}

The application will provide an api which can do a create, list and retrieve(single object) call for both shifts and employees.

Last of all the modules should only talk with each other via the \fullref{sec:API}.

All of these tests are done using Windows 10 on a Dell XPS 13, using the git bash terminal and the usage of MySQL as the primary database.

The assumption is made that the database exists where \texttt{root} is the username and password. The web framework used is the name of the database table.

All the code can be found at \url{https://github.com/jessielaf/modular_monolith}

Rails and Spring were not successful in the support of creating a modular monolith. Django on the other hand is the only framework that supports this type of architecture out of the box. This already gives the edge to Django as the framework to use. But what amplifies this choice is the amount of code that is needed to write in order for the test to work was minimal in comparison to the other frameworks. Django was also the only framework with build-in database migration generation. This allows the user to create migrations based on the model. This is very important because it eliminates human error when creating migrations by hand.

\subsection{Frontend}
\label{sec:Frontend}

Frontend is a very fast moving lane in software engineering. On october 8th 2010 \cite{angularJs} the first big frontend framework was published called AngularJs. This framework has been maintained by google and received a lot of traction. Three years later at Js ConfUS Jordan Walke of Facebook gave an introduction to React \cite{reactJs}. This changed the frontend world. Mainly because react was not a framework but a library. Which means that you are able to include it in your existing project whereas with Angular you solely have an Angular application. In February 2014 the last big javascript framework would be released called Vue \cite{vueJs}. Vue is often seen as the perfect blend between React and Angular. This is partly because Vue can be used as a library and a framework.

These three frameworks / libraries where chosen because they are the most used and the most loved by the javascript community \cite{allFrontendFrameworks}.

First off there is a need to define what should be in the api of each module in the frameworks. There is only one actual layer the modules should export and that is the service. A service is object which translates the rest api into an object api. They should export all the CRUD functionalities.

The scope of the test is that the frontend application should use our backend site created in \fullref{sec:BackendImplementation}. So the application should be able to do:

\begin{itemize}
    \item Create a employee
    \item List the employees
    \item Detail employee view
    \item List shifts
    \item Add shifts
    \item Detail shift view
\end{itemize}

All the code can be found at \url{https://github.com/jessielaf/modular_monolith}

From the implementations it is obvious that Angular is harder to implement than Vue or React. This is a combination of typescript and dependency injection which Angular uses. Vue and React on the other hand are really similar. But there are a few differences that makes Vue easier to use than React. The first one is the two way binding of Vue \cite{vueTwoWay}. React does not have this feature. What this means is that you have to write your own handler for every different input type. Another difference is that with Vue the router is included. Thus the Vue router is supported by the official team. React does not have a router build in. The best frontend framework for the modular monolith is Vue.

\section{Building the modular monolith}
\label{sec:BuildingModularMonolith}

The application consists of multiple modules and these modules need to be assembled before the application can be used. Assembling the modules should be easy on build when the application is being deployed but also when a developer is cloning the application. The first option is to create a command that will clone all the modules. This can be done per project and with a cli (command line interface) tool or with code inside of the application itself. The best option is the cli because this gives a common interface that can be used in multiple projects. The config for the modules and the projects will look the same. An example of such a system is a dependency managers. Managers such as \href{https://yarnpkg.com/en/}{Yarn} or \href{https://getcomposer.org/}{Composer}. These managers use a JSON file to define which versions are used of which dependencies or in this case modules. The config files for a modular monolith could be more stripped down because the requirements are less when comparing it to a full functional dependency manager.

There are two things all dependency managers have, a \textbf{name} and a \textbf{version}. What is unique to the modular monolith situation is that our name can be the same but where the module is \textbf{located} from can be different. Because there is a possibility that a specific module is required with a different version. This is all module level but there should also some project level settings. Such as the \textbf{directory} where the modules should be copied to.

\begin{verbatim}
    copy_dir: modules
    modules:
      employees:
        repo: git@github.com:jessielaf/employees-module
        version: master
      shifts:
        repo: git@github.com:jessielaf/shifts-module
        version: 1.0
\end{verbatim}

In this example the employees module will be retrieved from git@github.com:jessielaf/employees-module with the latest version and will be copied to \texttt{modules/employees}.

An example how the code could work is below:
\begin{verbatim}
    import yaml
    from git import Repo


    with open("example_config.yml", "r") as stream:
        config = yaml.safe_load(stream)
        modules = config['modules']
        copy_dir = config['copy_dir']

        for name, module in modules.items():
            repo = Repo.clone_from(module['repo'], f"{copy_dir}/{name}")
            repo.git.checkout(module['version'])
\end{verbatim}

Of course this is a very stripped down version of what can be. The best upside to this solution is that it can be the same for frontend and backend. The other option is to add the modules with the use of the framework. This makes it easier to use for new developers because they don't need to install a plugin or dependency that does this. But the code for the frontend and the backend needs to be managed and exists in the repositories. This makes it that the frontend config can be different from the backend one and it creates a harder to understand config. This is why the choice goes to the one config meets all option.

\section{Deployment lane}

As mentioned earlier the architecture that has been chosen for an application can make a big impact on the deployment lane. This is the reason why this section is chosen as an important piece for this research. But the choice of architecture came down to a solution where the deployment lane should not change or only change with building the application as explained in \fullref{sec:BuildingModularMonolith} this can be done with one commando. The deployment lane can thus stay virtually the same as before for EFFE.

\section{Implementing it in the application}

\subsection{Backend}

First of all the module api should be added. This can be done in \texttt{effe/api} and looks like this:
\begin{verbatim}
    from typing import Any
    from dataclasses import dataclass


    @dataclass
    class ModuleAPI:
        model: Any
        serializer: Any
        serializer_per_role: Any = None
\end{verbatim}

The first module that will be converted is shifts. This is because this is the biggest module. If this module can be converted all of them can be. The first thing to do is replace a direct link to the model to a link via the module. This means
\begin{verbatim}
    form shift.model import Shift
\end{verbatim}

Can be replaced with
\begin{verbatim}
    from shift.api import api as shift
\end{verbatim}

The use of the model looked like this:
\begin{verbatim}
    Shift.objects.all()
\end{verbatim}

And can be changed to:
\begin{verbatim}
    shift.model.objects.all()
\end{verbatim}

For the serializers the same principles apply.
\begin{verbatim}
    from shift.serializers.base import BaseSerializer
\end{verbatim}

Can be changed to:
\begin{verbatim}
    from shift.api import api as shift
\end{verbatim}

So when referring to the shift serializer:
\begin{verbatim}
    shifts = BaseSerializer()
\end{verbatim}

To:
\begin{verbatim}
    shifts = shift.serializer()
\end{verbatim}

The first problem that this created is that you cannot import a serializer into a model. This happens because python imports all classes even if it does not use one. Apparently this creates a circular dependency. When python has a circular dependency it gives a \texttt{ImportError: cannot import name} error. There is an open question on stackoverflow that has not been answered yet \cite{circularDependencyQuestion}.

The dependencies of the api need to be lazy loaded. There are two options on how to do this. The first one is to overwrite a function in the api. The ModuleApi would look like:
\begin{verbatim}
    from abc import abstractmethod


    class ModuleAPI:
        @abstractmethod
        def serializer(self):
            pass

        @abstractmethod
        def model(self):
            pass
\end{verbatim}

The api would look like this:
\begin{verbatim}
    from effe.api import ModuleAPI


    class Api(ModuleAPI):
        def serializer():
            from shift.serializers.api.base import BaseSerializer

            return BaseSerializer

        def model():
            from shift.models import Shift

            return Shift

    api = Api()
\end{verbatim}

The other method is string based imports. Where the ModuleApi would look like this:
\begin{verbatim}
    import importlib
    from typing import Dict


    class ModuleAPI:
        _model: str
        _model_package: str
        _serializer: str
        _serializer_per_role: Dict[str, str]

        def __init__(
            self,
            model_package: str,
            model: str,
            serializer: str,
            serializer_per_role: Dict[str, str] = {},
        ):
            self._model = model
            self._model_package = model_package
            self._serializer = serializer
            self._serializer_per_role = serializer_per_role

        def model(self):
            return getattr(importlib.import_module(self._model_package), self._model)

        def serializer(self):
            return importlib.import_module(self._serializer).BaseSerializer
\end{verbatim}

And the api:
\begin{verbatim}
    from effe.api import ModuleAPI

    api = ModuleAPI("shift.models", "Shift", "shift.serializers.base")
\end{verbatim}

The first options is better. This is because the model is imported directly this means that when renaming models or paths some IDE's will pick it up themselves. It is also more explicit and pylinters will pick up if a module cannot be imported.

Switching to the new api was easy until the user shift view needed to be changed. This view shows the the shifts of one user and the shift change requests of a user. This means that inside the shift serializer the shift change request serializer should be applied. But to make sure there are no circular dependencies this can't be done inside the \texttt{BaseSerializer} of the shift module. So there needs to be a function on which fields need to be serialized. Thus the api needs to be rewritten.

The django rest framework uses a private field \texttt{\_declared\_fields} that contains the nested serializers. The new api looks like this:
\begin{verbatim}
    from abc import abstractmethod
    from typing import Dict

    from rest_framework.serializers import Serializer


    class ModuleAPI:
        @abstractmethod
        def _serializer(self):
            pass

        @abstractmethod
        def model(self):
            pass

        def serializer(self, serializers: Dict[str, Serializer] = None):
            base_serializer = self._serializer()

            if serializers:
                for name, serializer in serializers.items():
                    base_serializer._declared_fields[name] = serializer
                    base_serializer.Meta.fields.append(name)

            return base_serializer
\end{verbatim}

In the shift overview the field shiftchangerequest\_set should be serialized by the shift change request serializer. The serializer looked like this:
\begin{verbatim}
    serializer_class = serializers.ShiftOverviewSerializer
\end{verbatim}

And with the modular monolith looks like this:
\begin{verbatim}
    serializer_class = Shift.serializer(
        {
            "shift_change_request": ShiftChange.serializer()(
                read_only=True, source="shiftchangerequest_set", many=True
            )
        }
    )
\end{verbatim}

The class \texttt{serializers.ShiftOverviewSerializer} can be removed because it is not used anywhere anymore.

In order to use the cli described in \fullref{sec:BuildingModularMonolith} cleaner, a project was made. It can be found at \url{https://github.com/jessielaf/modad}.

To implement this there needs to be a creation of the config in \texttt{modad.yml} which looks like this:
\begin{verbatim}
    dest: .
    modules:
        - name: shift
          repo: git@github.com:jessielaf/effe-shift
          version: master
\end{verbatim}

Then push the shift module to github at \url{https://github.com/jessielaf/effe-shift}. And remove the shift module from the base application and add shift to the .gitignore so github does not push the module to the base applications git. Now this command can be ran
\begin{verbatim}
    modad assemble
\end{verbatim}

Now the shift module is pulled from the github repository and in the application.

\subsection{Frontend}

The first module that was implemented in the backend was the shifts module. But EFFE uses Nuxt.js as a framework on top of Vue. Nuxt creates the routing via the folder structure. But this folder structure is decoupled from the business logic. There are two options to fix this:

\begin{itemize}
    \item \textbf{Adding two directories}: The API as described in \fullref{sec:API} needs to be changed so it will also handle multiple directories where it should copy to.
    \item \textbf{Rewriting it to only Vue}: As shown in \fullref{sec:Frontend} Vue is capable of having a modular monolith architecture. But to get this there needs to be a rewrite of the whole frontend application. There is a lot that can be refactored but there are also a lot of Nuxt.js functionalities the application uses.
\end{itemize}

Even though the second option ties more into the modular monolith architecture, the first option is better for a proof of concept for EFFE.

The config in \texttt{modad.yml} is needed to initialize our modular monolith is:
\begin{verbatim}
    dest:
        - src: pages
          dest: pages
        - src: module
          dest: modules
    modules:
        - name: shift
          repo: git@github.com:jessielaf/effe-ui-shift

\end{verbatim}

This will clone the most of the code into the \texttt{src/modules} and pages into \texttt{src/pages}. This is a one way street. The modules will now only be cloned. But by this logic the modules can also be created from this config. This makes it easy to use in development. Just load a module, change somethings and pull it back up. Thus the assembler also needs to dissemble the modules. This dissembler looks at the config and based on it, it creates a new folder with the changes made to the directories that are copied by the assembler.

The current shift module is intertwined with the change request, availability and shift market modules. The first action is to remove these from the shift module. Next up is the api for the shift module. Because Nuxt.js is used instead of only Vue the api looks a bit different:
\begin{verbatim}
    export default class Api {
        static create(axios, object, options = {}) {
            console.log(object, "not saved");
            console.error("Implement the create functionality");
        }

        static update(axios, id, object, options = {}) {
            console.log(object, "not saved");
            console.error("Implement the create functionality");
        }

        static retrieve(axios, id, options = {}) {
            console.log(id, "not retrieved");
            console.error("Implement the retrieve functionality");
        }

        static overview(axios, options = {}) {
            console.error("Implement the overview functionality");
        }
    }
\end{verbatim}

The same command as in the backend can be ran here:
\begin{verbatim}
    modad assemble
\end{verbatim}

The application works as it did before.
