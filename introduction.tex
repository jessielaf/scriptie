\chapter{Introduction}

\section{EFFE Planning}
EFFE Planning, hereinafter referred to as EFFE, relieves the pain of inefficient scheduling with their automatic scheduling system. EFFE has build software that prevents companies from making mistakes, lowers their cost and makes it easier for both the planner and employee. EFFE distinguishes themselves, from competition through our automated system and our intuitive user interface. The competition started ten years ago and is still focused on software, where the planner need to couple the employee with the shift. This is very time-inefficient. The basis of their software is build upon this system, while EFFE has their focus on a 100\% automated system. Automation is the future, and EFFE wants to become the leading player in this market.

\section{Motive}
EFFE as a company uses the SaaS (Software as a Service) model in order to comply to it’s expected growth. The basic SaaS model includes the basic application or MVP. This is in order to keep the application as abstract as possible. So that every company can connect their scheduling procedure to EFFE. But EFFE also wants to cater to the needs of bigger clients. This is why EFFE has created building blocks. More about the actual definition of building blocks can be found in \fullref{sec:TheProblem}

Building blocks are features that can be added/removed from the application. This can be done by the user or by EFFE. Examples of building blocks are white labeling, integration with frontend system and payrolling integration. These building blocks are not required when acquiring EFFE but can be added one by one.

\section{Intention}
\label{sec:Intention}

So the question is, how is EFFE going to implement these building blocks. There are a few requirements:
\begin{itemize}
	\item The building blocks need to be interchangeable. Meaning the same building block can be changed with another one that does the same job with slightly changed business logic.

	\item The building blocks should be able to do everything programming related. From if else to database calls.

	\item The building blocks can be either frontend or backend related

	\item Building block should be completely separate from the application (loosely coupled)
\end{itemize}
